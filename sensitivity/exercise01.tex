\documentclass[../main.tex]{subfiles}

\newcommand{\tema}{\textbf{SENSITIVITY ANALYSIS}}
%\printanswers
\begin{document}

\noindent NOMBRE COMPLETO:\hfill \enspace
Fecha:\hspace{1cm}/\hspace{1cm}/\hspace{1cm}
\vspace{-2mm}
\par\noindent\rule{\textwidth}{1pt}

%%%% MIDTERM HEADER  %%%%%%%%%%
{\noindent \footnotesize INSTRUCCIONES: Resolver los ejercicios que se presentan. Usa el método análitico. Para cada ejercicio deberás mostrar lo siguiente: \begin{itemize}
    \item Número de soluciones básicas.
    \item Todos los sistemas de ecuaciones lineales en formato matricial.
    \item Variables básicas y variables no-básicas.
    \item Clasificación de la solución: solución factible no-degenerada,solución factible degenerada o solución infactible.
    \item Valor de la función objetivo si la solución es factible.
    \item Solución óptima.
    \item Si la solución básica no se puede calcular,  indicar que el sistema no tiene solución. Asegurarse que ejecutaron las operaciones correctamente.
\end{itemize}

Este problema se evalúa en el rubro de \textbf{INVESTIGACIÓN}.

%%%%%%%%%%%%%%%%%%%%%%%%%%%%%%

% {\centering
% \includegraphics[scale=0.7]{final/rubrica.png}
% \par}
\vspace{3mm}
\par\noindent\rule{\textwidth}{1pt}
% \vspace{5mm}


\begin{questions}
\question % Gupta EXERCISES 6.6-10
Determinar los rangos sobre los cuales pueden cambiar los requerimientos de lado derecho en el siguiente problema


 \[\max Z = 3x_1 + 4x_2 + x_3 + 7x_4 \]

{\centering
sujeto a 

\sysdelim..%
\sysalign{r,r}%
\systeme[x_1x_2x_3x_4]%
{%
8x_1 + 3x_2 + 4x_3 + x_4  \leq 7,
2x_1 + 6x_2 + 1x_3 + 5x_4  \leq 3,
x_1 + 4x_2 + 5x_3 + 2x_4  \leq 8
}

\vspace{3mm}
$x_1, x_2, x_3, x_4 \geq 0$
\par}

\begin{solution}

La base del problema se compone de las variables $x_1, x_4,  s_3$, por lo tanto la solución óptima del problema es
\begin{flalign*}
  x_1 & = \nicefrac{16}{19} \\
  x_4 & = \nicefrac{5}{19} \\
  s_3 & = \nicefrac{126}{19}\\
  Z_{\max} & = \nicefrac{83}{19}
\end{flalign*}

La matriz inversa de la base del problema es

\[
  \bm{B^{-1}}  = %
\begin{bmatrix}
  \frac{5}{38} & -\frac{1}{38} & 0 \\[3mm]
  -\frac{1}{19} & \frac{4}{19} & 0 \\[3mm]
  -\frac{1}{38} & -\frac{15}{38} & 1 \\
\end{bmatrix}
\]

El cálculo de $\Delta b_i$ se realiza relizando las siguientes operaciones:

$\begin{bmatrix}
  \frac{5}{38} & -\frac{1}{38} & 0 \\[3mm]
  -\frac{1}{19} & \frac{4}{19} & 0 \\[3mm]
  -\frac{1}{38} & -\frac{15}{38} & 1 \\
\end{bmatrix}
\begin{bmatrix}
  7 + \Delta b_1\\[3mm] 3 \\[3mm] 8\\
\end{bmatrix}
\geq
\begin{bmatrix}
  0 \\[3mm] 0 \\[3mm] 0\\
\end{bmatrix}
$


$\begin{bmatrix}
  \frac{5}{38} & -\frac{1}{38} & 0 \\[3mm]
  -\frac{1}{19} & \frac{4}{19} & 0 \\[3mm]
  -\frac{1}{38} & -\frac{15}{38} & 1 \\
\end{bmatrix}
\begin{bmatrix}
  7 \\[3mm] 3 + \Delta b_2 \\[3mm] 8\\
\end{bmatrix}
\geq
\begin{bmatrix}
  0 \\[3mm] 0 \\[3mm] 0\\
\end{bmatrix}
$

$\begin{bmatrix}
  \frac{5}{38} & -\frac{1}{38} & 0 \\[3mm]
  -\frac{1}{19} & \frac{4}{19} & 0 \\[3mm]
  -\frac{1}{38} & -\frac{15}{38} & 1 \\
\end{bmatrix}
\begin{bmatrix}
  7 \\[3mm] 3 \\[3mm] 8 + \Delta b_3 \\
\end{bmatrix}
\geq
\begin{bmatrix}
  0 \\[3mm] 0 \\[3mm] 0\\
\end{bmatrix}
$



\end{solution}


\begin{solution}

    \begin{align*}
    -\frac{32}{5} & \leq \Delta b_1 \leq  5 \\[4mm]
    -\frac{5}{4}  &\leq \Delta b_2 \leq  \frac{84}{5} \\[4mm]
     -\frac{126}{19} & \leq \Delta b_3 \\
  \end{align*}
    
\end{solution}
    
\question % Gupta EXERCISES 6.6-13
Resolver el siguiente problema

\[\max Z = x_1 + 1.5x_2\]

{\centering
sujeto a 

\sysdelim..%
\sysalign{r,r}%
\systeme[x_1x_2]%
{%
2x_1  + 2x_2  \leq 160,
x_1  + 2x_2  \leq 120,
4x_1 + 2x_2 \leq 280
}

\vspace{3mm}
$x_1, x_2\geq 0$
\par}

\begin{solution}

  \begin{align*}
    x_1 & = 40\\
    x_2 & = 40 \\[2mm]
    Z_{\max} & = 100
  \end{align*}
\end{solution}

\begin{parts}
  \part ¿Cuáles son los valores del coeficiente $x_2$ en la función objetivo para que la solución siga siendo óptima?

  \begin{solution}

    $1 \leq c_2 \leq 2$    
  \end{solution}
  \part Determinar el rango óptimo para $c_1$

    \begin{solution}

   $ \frac{3}{4} \leq c_1 \leq \frac{3}{2}$
  \end{solution}
\end{parts}


\end{questions}
\end{document}