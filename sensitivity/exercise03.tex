\documentclass[../main.tex]{subfiles}

\newcommand{\tema}{\textbf{SENSITIVITY ANALYSIS}}
\begin{document}
\printanswers
\noindent NOMBRE COMPLETO:\hfill \enspace
Fecha:\hspace{1cm}/\hspace{1cm}/\hspace{1cm}
\vspace{-2mm}
\par\noindent\rule{\textwidth}{1pt}

%%%% MIDTERM HEADER  %%%%%%%%%%
{\noindent \footnotesize INSTRUCCIONES: Resolver los ejercicios que se presentan. Usa el método análitico. Para cada ejercicio deberás mostrar los siguiente: \begin{itemize}
    \item Número de soluciones báscias
    \item Todos los sistemas de ecuaciones lineales en formato matricial
    \item Variables básicas y variables no-básicas
    \item Clasificación de la solución: solución factible no-degenerada,solución factible degenerada o solución infactible
    \item Valor de la función objetivo si la solución es factible
    \item Solución óptima.
    \item Si la solución básica no se puede calcular,  indicar que el sistema no tiene solución. Asegurarse que ejecutaron las operaciones correctamente.
\end{itemize}


%%%%%%%%%%%%%%%%%%%%%%%%%%%%%%

% {\centering
% \includegraphics[scale=0.7]{final/rubrica.png}
% \par}
\vspace{3mm}
\par\noindent\rule{\textwidth}{1pt}
% \vspace{5mm}

% ============================
% ADD NEW VARIABLE
% ============================
\begin{questions}
\question % Gupta EXERCISES 6.6-24

A manufacturing company produces two products, each of which requires stamping, assembly and painting operations. Total productive capacity by operation if it were devoted solely to one product or the other is

{\centering
  \begin{tabular}{ccc}
    \toprule
    Operation& \multicolumn{2}{c}{Productive Capacity (Units / week)} \\
    \cmidrule{2-3}
             &Product A& Product B \\
    \midrule
    Stamping & 500 & 700\\
    Assembly & 650 & 330\\
    Painting & 450 & 700\\
       \bottomrule
  \end{tabular}
  \par}

Demand for the two products is unlimited and the profits on A and B are \$ 20 and \$ 15 respectively. The company wants to expand its product line. Its marketing manager has determined that there is an unlimited market for a third product C. The productive capacity for
this product is

{\centering
  \begin{tabular}{cc}
    \toprule
    Operation& Product C (Units / week) \\
    \midrule
    Stamping & 200 \\
    Assembly & 180 \\
    Painting & 120 \\
       \bottomrule
  \end{tabular}
  \par}

The unit profit for product C is estimated to be \$ 16. Should the company extend its product line by including product C ?


\begin{solution}{}
  Las variables de decisión son las unidades a producir de A y B, denotadas con las variables $x_1$ y $x_2$ respectivamente.  El modelo inicial es el siguiente
  
  \[\max Z = 20x_1 + 15x_2  \]

  \begin{align*}
    \frac{x_1}{500} + \frac{x_2}{700}   &\leq 1\\[3mm]
\frac{x_1}{650} + \frac{x_2}{330}  &\leq 1\\[3mm]
    \frac{x_1}{450} + \frac{x_2}{700}   &\leq 1\\[3mm]
    x_1, x_2 &\geq 0
  \end{align*}
  

La solución óptima para este problem se muestra en la siguiente tabla

{\centering
  \begin{tabular}{rrrrrrrr}
    \toprule
    &&20&	15&	0&	0&	0&	\\
    \midrule
    $c_B$&	basis&	$x_1$&	$x_2$&	$s_1$&	$s_2$&	$s_3$&	\textbf{b}\\
    \midrule
0&	$s_1$&	0&	0&	1&	-0.069984&	-0.852&	0.078\\
15&	$x_2$&	0&	1&	0&	489.885808&	-339.152&	150.734\\
    20&	$x_1$&	1&	0&	0&	-314.927&	668.026&	353.099\\
    \midrule
	&$Z_j$&	20&	15&	0&	1049.755&	8273.246&	9323.00\\
    &$c_j - Z_j$&	0&	0&	0&	-1049.755&	-8273.246&\\
    \bottomrule
  \end{tabular}
  \par}

\vspace{3mm}

La matriz base $\pmb{B}$ se compone de los vectores correspondientes a \textbf{las variables básicas} $s_1, x_2, x_1$ en el problema en su forma estándar como se ve en la tabla siguiente

\newpage
{\centering
  \begin{tabular}{rrrrrrrr}
    \toprule
    &&20&	15&	0&	0&	0&	\\
    \midrule
    $c_B$&	basis&	$x_1$&	$x_2$&	$s_1$&	$s_2$&	$s_3$&	\textbf{b}\\
    \midrule
0&	$s_1$&	\nicefrac{1}{500}&\nicefrac{1}{700}	&	1&	0&	0&	1\\
0&	$s_2$&	\nicefrac{1}{650}&	\nicefrac{1}{330}&	0&	1&	0&	1\\
    0&	$s_3$&	\nicefrac{1}{450}&	\nicefrac{1}{700}&	0&	0&	1&	1\\
    \bottomrule
  \end{tabular}
  \par}

la matriz base es $\pmb{B} =
\begin{bmatrix}
  1 & \nicefrac{1}{700} & \nicefrac{1}{500} \\
  0 & \nicefrac{1}{330} & \nicefrac{1}{650} \\
  0 & \nicefrac{1}{700} & \nicefrac{1}{450} \\
\end{bmatrix}
$
y la matriz inversa de $\pmb{B}$ es $\pmb{B^{-1}} =
\begin{bmatrix}
1&	- 429/6130&	- 522/613 \\
0&	300300/613 &	-207900/613 \\
0&	-193050/613& 	409500/613 \\
\end{bmatrix}
$

Los vectores que componen a $\pmb{B}$ son $ [\pmb{P_3}, \pmb{P_2}, \pmb{P_1}]$, esto es,  $\pmb{P_3} =
\begin{bmatrix}
  1\\
  0\\
  0\\
\end{bmatrix},\;
\pmb{P_2} =
\begin{bmatrix}
  \nicefrac{1}{700}\\
  \nicefrac{1}{330}\\
  \nicefrac{1}{700}\\
\end{bmatrix},\;
\pmb{P_1} =
\begin{bmatrix}
  \nicefrac{1}{500}\\
  \nicefrac{1}{650}\\
  \nicefrac{1}{450}\\
\end{bmatrix}
$

Se debe recordar que \[ \overline{c_j} = c_j - Z_j = c_j - c_B \cdot B^{-1} \cdot P_j\] de la expresión anterior se define que \[\pmb{\pi} = c_B \cdot B^{-1} \]

El cálculo de los multiplicadores simplex es $\pi = c_B \cdot B^{-1} = \left[0.00, 1049.76, 8273.25\right]$, entonces si $x_3$ es la nueva variable que se va a agregar al problema correspondiente al producto C tenemos que
$Z_3 = \pi P_3 = \begin{bmatrix}
  0,& 1049.755,& 8273.246\\
\end{bmatrix}
\begin{bmatrix}
  \frac{1}{200}\\[2mm]
  \frac{1}{180}\\[2mm]
  \frac{1}{120}\\
\end{bmatrix}
 = 74.78
 $
 
por lo tanto $\overline{c_3} = c_3 - Z_3 = 16 - 74.78 = -58.78$ esto nos da $\overline{c_3} \leq 0$, \textbf{el problema es de maximización} entonces la solución sigue siendo óptima y por lo tanto no vale la pena incluir el producto C al modelo ya que no aporta ninguna mejora. 
\end{solution}


\end{questions}
\end{document}