\documentclass[../main.tex]{subfiles}

% \newcommand{\tema}{\textbf{SENSITIVITY ANALYSIS}}
\newcommand{\tema}{\textbf{FINAL EXAM}}
%\printanswers

\begin{document}

\noindent NOMBRE COMPLETO:\hfill \enspace
Fecha:\hspace{1cm}/\hspace{1cm}/\hspace{1cm}
\vspace{-2mm}
\par\noindent\rule{\textwidth}{1pt}

%%%% MIDTERM HEADER  %%%%%%%%%%
{\noindent \footnotesize INSTRUCCIONES: Resolver los ejercicios que se presentan. Usa el método análitico. Para cada ejercicio deberás mostrar lo siguiente: \begin{itemize}
    \item Número de soluciones básicas.
    \item Todos los sistemas de ecuaciones lineales en formato matricial.
    \item Variables básicas y variables no-básicas.
    \item Clasificación de la solución: solución factible no-degenerada,solución factible degenerada o solución infactible.
    \item Valor de la función objetivo si la solución es factible.
    \item Solución óptima.
    \item Si la solución básica no se puede calcular,  indicar que el sistema no tiene solución. Asegurarse que ejecutaron las operaciones correctamente.
\end{itemize}

Este problema se evalúa en el rubro de \textbf{INVESTIGACIÓN}.

%%%%%%%%%%%%%%%%%%%%%%%%%%%%%%

% {\centering
% \includegraphics[scale=0.7]{final/rubrica.png}
% \par}
\vspace{3mm}
\par\noindent\rule{\textwidth}{1pt}
% \vspace{5mm}

% ============================
% ADD NEW VARIABLE
% ============================
\begin{questions}
% \question % Gupta EXERCISES 6.6-27
% A manufacturer can produce four products using three resources namely labour, raw material and machine time. The unit contributions and the technological constraints are given below in the form of an L.P.P. :

% \[\max Z = 3x_1 + 3x_2 + 4x_3 + 7x_4 \]

% {\centering
% subject to

% \sysdelim..%
% \sysalign{r,r}%
% \systeme[x_1x_2x_3x_4]%
% {%
% x_1 + x_2 + x_3 + x_4  \leq 9,
% 5x_1 + 3x_2 + 2x_3 + x_4  \leq 60,
% x_1 + 3x_2 + 5x_3 + 8x_4  \leq 50
% }

% \vspace{3mm}
% $x_1, x_2, x3, x4 \geq 0$
% \par}

% \begin{parts}
% \part Find the optimal product mix.
% \part Suppose a fifth product represented by variable $x_8$ is to be included in the optimal product  mix, what should be its unit contribution if the resource requirements for this product are $\frac{7}{4}$ units, 3 units and 7 units of labour, material and machine-time respectively ?
% \end{parts}

% \begin{solution}{}
%   \[\max Z = 3x_1 + 3x_2 + 4x_3 + 7x_4 + x_5 \]

% {\centering
% subject to

% \sysdelim..%
% \sysalign{r,r}%
% \systeme[x_1x_2x_3x_4x_5]%
% {%
% x_1 + x_2 + x_3 + x_4  + \nicefrac{7}{4}x_5\leq 9,
% 5x_1 + 3x_2 + 2x_3 + x_4 + 3x_5 \leq 60,
% x_1 + 3x_2 + 5x_3 + 8x_4  + 7x_5\leq 50
% }

% \vspace{3mm}
% $x_1, x_2, x3, x4, x_5 \geq 0$
% \par}
% \end{solution}

\question
Para el siguiente Modelo de Programación Lineal

\[\max  Z = 3x_1 + 4x_2 + 6x_3 + 10x_4\]
sujeto a
\begin{align*}
  x_1 + x_2 + x_3 + x_4 & \leq 12 \\
  6x_1 + 4x_2 + 2x_3 + x_4 & \leq 90 \\
  2x_1 + 4x_2 + 9x_3 + 10x_4 & \leq 70 \\[5mm]
  x_1, x_2, x_3, x_4 & \geq 0
\end{align*}

encontrar los rangos de los requerimientos del lado derecho que garantizan una solución factible.

\begin{solution}


La base del problema se compone de las variables $x_1,  s_2, x_4$, por lo tanto la solución óptima del problema es
\begin{flalign*}
  x_1 & = \nicefrac{25}{4} \\
  s_2 & = \nicefrac{187}{4}\\
  x_4 & = \nicefrac{23}{4} \\
  Z_{\max} & = \nicefrac{305}{4}
\end{flalign*}
  
  La matriz inversa de la base $\bm{B^{-1}}$ es

  \[
    \bm{B^{-1}} = %
    \begin{bmatrix}
      5/4&0&-1/8   \\[2mm]
      - 29/4&1&5/8\\[2mm]  
      - 1/4&0&1/8   
    \end{bmatrix}
  \]


  El cálculo de $\Delta b_i$ se realiza relizando las siguientes operaciones:

$\begin{bmatrix}
        5/4&0&-1/8   \\[2mm]
        - 29/4&1&5/8\\[2mm]  
        - 1/4&0&1/8   
    \end{bmatrix}
\begin{bmatrix}
  12 + \Delta b_1\\[2mm] 90 \\[2mm] 70\\
\end{bmatrix}
\geq
\begin{bmatrix}
  0 \\[2mm] 0 \\[2mm] 0\\
\end{bmatrix}
$

$\begin{bmatrix}
        5/4&0&-1/8   \\[2mm]
        - 29/4&1&5/8\\[2mm]  
        - 1/4&0&1/8   
\end{bmatrix}
\begin{bmatrix}
  12 \\[2mm] 90 + \Delta b_2 \\[2mm] 70\\
\end{bmatrix}
\geq
\begin{bmatrix}
  0 \\[2mm] 0 \\[2mm] 0\\
\end{bmatrix}
$

$\begin{bmatrix}
        5/4&0&-1/8   \\[2mm]
        - 29/4&1&5/8\\[2mm]  
        - 1/4&0&1/8   
\end{bmatrix}
\begin{bmatrix}
  12 \\[2mm] 90  \\[2mm] 70 + \Delta b_3\\
\end{bmatrix}
\geq
\begin{bmatrix}
  0 \\[2mm] 0 \\[2mm] 0\\
\end{bmatrix}
$
\end{solution}

\begin{checkboxes}
  \correctchoice  $ -5 \leq \delta_1 \leq \frac{187}{29};\quad \delta_2 \geq -\frac{187}{4};\quad -46 \leq \delta_3 \leq 50 $
  \choice $-5 \leq \delta_1 \leq \frac{187}{49};\quad  \delta_2 \geq -\frac{187}{6};\quad -46 \leq \delta_3 \leq 70 $
  \choice $-5 \leq \delta_1 \leq \frac{187}{19};\quad  \delta_2 \geq -\frac{187}{4};\quad -26 \leq \delta_3 \leq 50 $
  \choice $-2 \leq \delta_1 \leq \frac{187}{29};\quad  \delta_2 \geq -\frac{190}{4};\quad -46 \leq \delta_3 \leq 60 $
\end{checkboxes}


\end{questions}
\end{document}