\documentclass[../main.tex]{subfiles}

\newcommand{\tema}{\textbf{SENSITIVITY ANALYSIS}}
%\printanswers
\begin{document}

\noindent NOMBRE COMPLETO:\hfill \enspace
Fecha:\hspace{1cm}/\hspace{1cm}/\hspace{1cm}
\vspace{-2mm}
\par\noindent\rule{\textwidth}{1pt}

%%%% MIDTERM HEADER  %%%%%%%%%%
{\noindent \footnotesize INSTRUCCIONES: Resolver los ejercicios que se presentan. Usa el método análitico. Para cada ejercicio deberás mostrar los siguiente: \begin{itemize}
    \item Número de soluciones báscias
    \item Todos los sistemas de ecuaciones lineales en formato matricial
    \item Variables básicas y variables no-básicas
    \item Clasificación de la solución: solución factible no-degenerada,solución factible degenerada o solución infactible
    \item Valor de la función objetivo si la solución es factible
    \item Solución óptima.
    \item Si la solución básica no se puede calcular,  indicar que el sistema no tiene solución. Asegurarse que ejecutaron las operaciones correctamente.
\end{itemize}


%%%%%%%%%%%%%%%%%%%%%%%%%%%%%%

% {\centering
% \includegraphics[scale=0.7]{final/rubrica.png}
% \par}
\vspace{3mm}
\par\noindent\rule{\textwidth}{1pt}
% \vspace{5mm}

% ============================
% CHANGES IN THE COEFFICIENTS OF THE CONSTRAINTS
% ============================
\begin{questions}
\question % Gupta EXERCISES 6.6-28

Considere el siguiente problema

\[ \max Z = 3x_1  + 2x_2 + 5x_3\]


{\centering
  subject to

  \sysdelim..%
  \sysalign{r,r}%
  \systeme[x_1x_2x_3]%
  {%
    x_1 + 2x_2 + x_3 \leq 430,
    3x_1 + 2x_3 \leq 460,
    x_1 + 4x_2 \leq 420
  }

  \vspace{3mm}

  $x_1, x_2, x_3 \geq 0$
  \par}


Encuentre la solución óptima. Suponga que los coeficientes del lado izquierdo (coeficientes tecnológicos) de  la columna $x_1$ cambian de $[1 \;,\; 3 \;,\; 1]^{T}$ a $[1 \;,\; 1 \;,\; 6]^{T}$ en la matriz inicial y los coeficientes de $x_2$ and $x_3$ en la función objetivo cambian de $[2\;, \; 5]$ a $[1\;,\; 3]$ encuentre la nueva solución óptima.
\vspace{5mm}


\begin{solution}{}

  The optimal solution to this problem is given by the following table:

{\centering
  \begin{tabular}{crrrrrrrr}
    \toprule
    &$c_j$&3&2&5&0&0&0&\\
    \midrule
    $c_B$&Basis&$x_1$&$x_2$&$x_3$&$s_1$&$s_2$&$s_3$&b\\
    \midrule
    2&$x_2$&\nicefrac{-1}{4}&1&0&\nicefrac{1}{2}&\nicefrac{-1}{4}&0&100\\
    5&$x_3$&\nicefrac{3}{2}&0&1&0&\nicefrac{1}{2}&0&230\\
    0&$s_3$&2&0&0&$-2$&1&1&20\\
    \midrule
    $Z_j = \sum c_Ba_{ij}$& &7&2&5&1&2&0&\\
    $\overline{c_j} = c_j - Z_j$& &$-4$&0&0&$-1$&$-2$&0&\\
    \bottomrule
  \end{tabular}
  \par}

  Las particiones del problema (en forma estándar) son las siguientes:
  
  {\centering
    \begin{tabular}{rrrrrr}
      \toprule
      $\pmb{P_1}$	&$\pmb{P_2}$&	$\pmb{P_3}$&	$\pmb{P_4}$&	$\pmb{P_5}$&	$\pmb{P_6}$\\
      \midrule
      $x_1$	&$x_2$&	$x_3$&	$s_1$&	$s_2$&	$s_3$\\
      \midrule
      1&	2&	1&	1&	0&	0\\
      3&	0&	2&	0&	1&	0\\
      1&	4&	0&	0&	0&	1\\
      \bottomrule
    \end{tabular}

  \par}

De la tabla óptima se sabe que la base está compuesta por $x_2, x_3, s_3$, por lo tanto las particiones que forman la  matriz base son $\pmb{P_2}, \pmb{P_3}, \pmb{P_6}$ como se muestra a continuación:

{\centering
    \begin{tabular}{rrrrrr}
      \toprule
      $\pmb{P_1}$	&\cellcolor{blue!30}$\pmb{P_2}$&	\cellcolor{blue!30}$\pmb{P_3}$&	$\pmb{P_4}$&	$\pmb{P_5}$&	\cellcolor{blue!30}$\pmb{P_6}$\\
      \midrule
      $x_1$	&\cellcolor{blue!30}$x_2$&	\cellcolor{blue!30}$x_3$&	$s_1$&	$s_2$&	\cellcolor{blue!30}$s_3$\\
      \midrule
      1&	\cellcolor{blue!30}2&	\cellcolor{blue!30}1&	1&	0&	\cellcolor{blue!30}0\\
      3&	\cellcolor{blue!30}0&	\cellcolor{blue!30}2&	0&	1&	\cellcolor{blue!30}0\\
      1&	\cellcolor{blue!30}4&	\cellcolor{blue!30}0&	0&	0&	\cellcolor{blue!30}1\\
      \bottomrule
    \end{tabular}

  \par}

Entonces $\pmb{B} =
\begin{bmatrix}
  2& 1& 0\\
  0& 2& 0\\
  4& 0& 1\\
\end{bmatrix}
$
 y la matriz inversa de la base es $\pmb{B^{-1}} =
 \begin{bmatrix}
   \nicefrac{1}{2} & \nicefrac{-1}{4} & 0\\
   0 & \nicefrac{1}{2} & 0\\
   -2 & 1 & 1\\
\end{bmatrix}
$

Al hacer el cambio en los coeficientes de las restricciones correspondientes a la variable $x_1$ los valores del nuevo vector $\pmb{P_1} =
\begin{bmatrix}
   1\\        
 1\\        
 6\\
\end{bmatrix}
$ en la tabla simplex óptima se calculan como \[ \overline{ \pmb{P_1}} = B^{-1} \cdot P_1 =%
  \begin{bmatrix}
   \nicefrac{1}{2} & \nicefrac{-1}{4} & 0\\
   0 & \nicefrac{1}{2} & 0\\
   -2 & 1 & 1\\
 \end{bmatrix} %
 \begin{bmatrix}
   1\\        
 1\\        
 6\\
\end{bmatrix}%
=
\begin{bmatrix}
  \nicefrac{1}{4}   \\
  \nicefrac{1}{2}   \\
  5 \\
\end{bmatrix}
\]

Al incluir el nuevo vector $ \overline{\pmb{P_1}} = \begin{bmatrix}
  \nicefrac{1}{4}   \\
  \nicefrac{1}{2}   \\
  5 \\
\end{bmatrix}$ en la tabla óptima y los nuevos valores de la función objetivo de $x_2, x_3 = [1 \;,\; 3]$ la nueva tabla simplex es


{\centering
  \begin{tabular}{rrrrrrrrr}
    \toprule
    &$c_j$	&3	&\cellcolor{blue!30}1&	\cellcolor{blue!30}3&	0&	0&	0&\\
    \midrule
    $c_B$&Basis&\cellcolor{blue!30}$x_1$&$x_2$&$x_3$&$s_1$&$s_2$&$s_3$&$\pmb{b}$\\
      \midrule
    1&	$x_2$& \cellcolor{blue!30}\nicefrac{1}{4}&   	1&	0&	   \nicefrac{1}{2}&   	\nicefrac{-1}{4}&   	0&	100\\
    3&	$x_3$&	   \cellcolor{blue!30}\nicefrac{1}{2}&   	0&	1&	0&	   \nicefrac{1}{2}&   	0&	230\\
    0&	$s_3$&	\cellcolor{blue!30}5&	0&	0&	-2&	1&	1&	20\\
      \midrule
	&$Z_j$&	\nicefrac{7}{4}&	1&	3&	\nicefrac{1}{2}&	\nicefrac{5}{4}&	0&	790\\
      &$c_j - Z_j$&	\nicefrac{5}{4}&	0&	0&	\nicefrac{-1}{2}&	\nicefrac{-5}{4}&	0&\\
      \toprule
  \end{tabular}
\par}

Se aplica algoritmo simplex sobre la tabla simplex anterior porque $\overline{c_1} > 0$, por lo que la solución actual deja de ser óptima, recordemos que el problema es de tipo maximización entonces se requiere que $c_j - Z_j \leq 0$. Aplicando el algoritmo simplex la variable $x_1$ es la nueva variable que entra a la base y la variable que deja la base es $s_3$.

La nueva tabla simplex con la solución óptima es:

{\centering
  \begin{tabular}{crrrrrrrr}
    \toprule
    &$c_j$&3&1&3&0&0&0&\\
    \midrule
    $c_B$&Basis&$x_1$&$x_2$&$x_3$&$s_1$&$s_2$&$s_3$&b\\
    \midrule
    1&$x_2$&0&1&0&\nicefrac{3}{5}&\nicefrac{-3}{10}&\nicefrac{-1}{20}&99\\
    3&$x_3$&0&0&1&\nicefrac{1}{5}&\nicefrac{2}{5}&\nicefrac{-1}{10}&228\\
    3&$x_1$&1&0&0&\nicefrac{-2}{5}&\nicefrac{1}{5}&\nicefrac{1}{5}&4\\
    \midrule
    $Z_j = \sum c_Ba_{ij}$& &7&2&5&1&2&0&\cellcolor{yellow}{795}\\
    $\overline{c_j} = c_j - Z_j$& &$-4$&0&0&$-1$&$-2$&0&\\
    \bottomrule
  \end{tabular}
  \par}

note que $c_j - Z_j \leq 0$ por lo que la solución es óptima.
\end{solution}

% ========================================
\question % Gupta EXERCISES 6.6-32
Considere el siguiente modelo de programación lineal.

\[ \max Z = 20x_1 + 10x_2 \]

{\centering
  subject to

  \vspace{3mm}

  \sysdelim..%
  \sysalign{r,r}%
  \systeme[x_1x_2]%
  {%
    x_1 + 2x_2  \leq 40,
    3x_1 + 2x_2 \leq 60
  }

  \vspace{3mm}

  $x_1, x_2 \geq 0$
  \par}


Encuentre la solución óptima. Si la columna $x_2$ cambia de  %
$
\begin{bmatrix}
  2\\
  2\\
\end{bmatrix}
$
a %
$
\begin{bmatrix}
  2\\
  1\\
\end{bmatrix}
$

Encuentre la nueva solución óptima

\begin{solution}{}
  La tabla óptima es la siguiente:
  
  {\centering
    \begin{tabular}{rrrrrrr}
      \toprule
      &$c_j$&	20&	10&	0&	0& \\
      \midrule
      $c_B$&	Basis&	$x_1$& $x_2$&	$s_1$&	$s_2$&	\textbf{b} \\
      \midrule
0&	$s_1$&	0&	   \nicefrac{4}{3}&   	1&	\nicefrac{-   1}{3}&   	20\\
20&	$x_1$&	1&	   \nicefrac{2}{3}&   	0&	   \nicefrac{1}{3}&   	20\\
	&$Z_j$&	20&	  \nicefrac{40}{3}&   	0&	  \nicefrac{20}{3}&   	\cellcolor{yellow}400\\
      &$c_j - Z_j$&	0&	\nicefrac{-10}{3}&   	0&	\nicefrac{-20}{3}&   \\
      \toprule
    \end{tabular}
  \par}	

 La matriz $\pmb{B^{-1}} =
 \begin{bmatrix}
   1 & \nicefrac{-1}{3}\\   
   0 &  \nicefrac{1}{3} \\
 \end{bmatrix}
 $

 El vector $\overline{\pmb{P_2}} = B^{-1} \cdot P_2 = %
  \begin{bmatrix}
   1 & \nicefrac{-1}{3}\\   
   0 &  \nicefrac{1}{3} \\
 \end{bmatrix}\cdot
 \begin{bmatrix}
   2\\
   1\\
 \end{bmatrix} =
 \begin{bmatrix}
   \nicefrac{5}{3}\\
   \nicefrac{1}{3}\\
 \end{bmatrix}
 $
 
Calculamos $\pmb{\pi} = c_B \cdot B^{-1} = %
\begin{bmatrix}
  0, & 20\\
\end{bmatrix} 
\begin{bmatrix}
   1 & \nicefrac{-1}{3}\\   
   0 &  \nicefrac{1}{3} \\
 \end{bmatrix} =
 \begin{bmatrix}
   0, & \nicefrac{20}{3} \\
 \end{bmatrix}
 $
 
 El valor de $\overline{c_2} = c_2 - Z_2 = c_2 - \pmb{\pi}\cdot P_2 = 10 - [0 \;,\; \nicefrac{20}{3}] \cdot
 \begin{bmatrix}
   2\\
   1\\
 \end{bmatrix} = 10 - \nicefrac{20}{3} = \nicefrac{10}{3}
 $

 {\centering
    \begin{tabular}{rrrrrrr}
      \toprule
      &$c_j$&	20&	10&	0&	0& \\
      \midrule
      $c_B$&	Basis&	$x_1$& $x_2$&	$s_1$&	$s_2$&	\textbf{b} \\
      \midrule
0&	$s_1$&	0&	   \cellcolor{blue!30}\nicefrac{5}{3}&   	1&	\nicefrac{-   1}{3}&   	20\\
      20&	$x_1$&	1&	   \cellcolor{blue!30}\nicefrac{1}{3}&   	0&	   \nicefrac{1}{3}&   	20\\
      \midrule
	&$Z_j$&	20&	  \nicefrac{20}{3}&   	0&	  \nicefrac{20}{3}&   	\cellcolor{yellow}400\\
      &$c_j - Z_j$&	0&	\cellcolor{cyan!30}\nicefrac{10}{3}&   	0&	\nicefrac{-20}{3}&   \\
      \toprule
    \end{tabular}
  \par}	

 Como $\overline{c_2} > 0$ la solución deja de ser óptima (problema de maximización), se aplica método simplex con variable $x_2$ como variable que entra y $s_1$ como variable que sale.

 La solución óptima al problema se muestra en la siguiente tabla simplex con $x_2$ en la base.

 {\centering
    \begin{tabular}{rrrrrrr}
      \toprule
      &$c_j$&	20&	10&	0&	0& \\
      \midrule
      $c_B$&	Basis&	$x_1$& $x_2$&	$s_1$&	$s_2$&	\textbf{b} \\
      \midrule
10&	$x_2$&	0&	   1&   	\nicefrac{3}{5}&	\nicefrac{-   1}{5}&   	12\\
      20&	$x_1$&	1&	   0&   	\nicefrac{-1}{5}&	   \nicefrac{2}{5}&   	16\\
      \midrule
	&$Z_j$&	20&	  10&   	2&	  6&   	\cellcolor{yellow}440\\
      &$c_j - Z_j$&	0&	0&   	$-2$&	$-6$&   \\
      \toprule
    \end{tabular}
  \par}	

 
\end{solution}
\end{questions}
\end{document}