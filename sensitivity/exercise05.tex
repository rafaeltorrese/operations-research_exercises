\documentclass[../main.tex]{subfiles}

\newcommand{\tema}{\textbf{SENSITIVITY ANALYSIS}}
\begin{document}
\printanswers
\noindent NOMBRE COMPLETO:\hfill \enspace
Fecha:\hspace{1cm}/\hspace{1cm}/\hspace{1cm}
\vspace{-2mm}
\par\noindent\rule{\textwidth}{1pt}

%%%% MIDTERM HEADER  %%%%%%%%%%
{\noindent \footnotesize INSTRUCCIONES: Resolver los ejercicios que se presentan. Usa el método análitico. Para cada ejercicio deberás mostrar lo siguiente: \begin{itemize}
    \item Número de soluciones básicas.
    \item Todos los sistemas de ecuaciones lineales en formato matricial.
    \item Variables básicas y variables no-básicas.
    \item Clasificación de la solución: solución factible no-degenerada,solución factible degenerada o solución infactible.
    \item Valor de la función objetivo si la solución es factible.
    \item Solución óptima.
    \item Si la solución básica no se puede calcular,  indicar que el sistema no tiene solución. Asegurarse que ejecutaron las operaciones correctamente.
\end{itemize}

Este problema se evalúa en el rubro de \textbf{INVESTIGACIÓN}.

%%%%%%%%%%%%%%%%%%%%%%%%%%%%%%

% {\centering
% \includegraphics[scale=0.7]{final/rubrica.png}
% \par}
\vspace{3mm}
\par\noindent\rule{\textwidth}{1pt}
% \vspace{5mm}

% ============================
% ADDITION OF A NEW CONSTRAINT
% ============================
\begin{questions}
\question % Gupta EXERCISES 6.6-41

Consider the following table which represents an optimal solution to some L.P.P.:

{\centering
  \begin{tabular}{crrrrrrrrrr}
    \toprule
    &$c_j$&2&4&1&3&2&0&0&0&\\
    \midrule
    $c_B$&Basis&$x_1$&$x_2$&$x_3$&$x_4$&$x_5$&$x_6$&$x_7$ &$x_8$& b\\
    \midrule
    2&$x_1$&1&0&0&$-1$&0&\nicefrac{1}{2}&\nicefrac{1}{5}&$-1$&3\\
    4&$x_2$&0&1&0&2&1&$-1$&0&\nicefrac{1}{2}&1\\
    1&$x_3$&0&0&1&$-1$&$-2$&5&$\nicefrac{-3}{10}$&2&7\\
    \midrule
    %$Z_j = \sum c_Ba_{ij}$& &7&2&5&1&2&0&\\
    &$\overline{c_j} = c_j - Z_j$ &0&0&0&$-2$&0&$-2$&$\nicefrac{-1}{10}$&$-2$&17\\
    \bottomrule
  \end{tabular}
  \par}

If the additional constraint $2x_1 + 3x_2 - x_3 + 2x_4 + 4 x_5 \leq 5$ is annexed to the system, will there be any change in the optimal solution? Justify your answer.


\begin{solution}{}
  Sustituyendo los valores de $x_1, x_2, x_3$ en la restricción que se desea agregar se tiene que \[ 2(3) + 3(1) - 7+ 0 + 0 = 6 + 3 - 7 + 0 +0 = 2 \]

  Los valores de la solución óptima satisfacen la nueva restricción ($2 \leq 5$) por lo tanto no hay cambio en el sistema
\end{solution}
\vspace{5mm}

\question % Gupta EXERCISES 6.6-43
A firm produces three items A, B, and C and requires two types of resources --man hours and raw matrial. The following L.P. problem has been formulated to determine the optimum production schedule that maximizes the total profit:

\[ \max Z = 3y_1 + y_2 + 5y_3\]

{\centering
  subject to
  \vspace{2mm}

  \sysdelim..%
  \sysalign{r,r}%
  \systeme[y_1y_2y_3]%
  {%
    6y_1 + 3y_2 + 5y_3 \leq 45@ (man-hours),
    3y_1 + 4y_2 + 5y_3 \leq 30@ (\text{raw material})
  }
  \vspace{2mm}

  $y_1, y_2, y_3 \geq 0$
  \par}

\vspace{3mm}

where $y_1, y_2, y_3$ are the number of items A, B, and C. Find the optimal solution.

\begin{solution}{}
  The optimal table is

  {\centering
    \begin{tabular}{rrrrrrrr}
    \toprule
    &$c_j$&	3&	1&	5&	0&	0&\\
    \midrule
    $c_B$&	Basis&	$y_1$&	$y_2$&	$y_3$&	$s_1$&	$s_2$&	$\pmb{b}$\\
    \midrule
0&	$s_1$&	3&	-1&	0&	1&	-1&	15\\
5&	$y_3$&	\nicefrac{3}{5}&	\nicefrac{4}{5}&	1&	0&	\nicefrac{1}{5}&	6\\
\midrule
	&$Z_j$&	3&	4&	5&	0&	1&	30\\
	&$c_j- Z_j$&	0&	-3&	0&	0&	-1&\\
        \toprule
  \end{tabular}
  \par}

\end{solution}
\vspace{3mm}

\begin{parts}
  \part Find the range on the unit profit of product A. If $c_1 = 4$, what is the optimal solution?
  \begin{solution}{}
    $c_1$ is the coefficient of the non-basic variable $y_1$. Then, \[ \overline{c_1} = c_1 - Z_j = 4 -
      \begin{bmatrix}
        0, & 5
      \end{bmatrix}
      \begin{bmatrix}
        3\\
        \nicefrac{3}{5}
      \end{bmatrix} = 4 - 3 = 1
    \]
    The solution is non-optimal, then, it is necessary to perform the simplex algorithm on the new table

    {\centering
    \begin{tabular}{rrrrrrrrr}
    \toprule
    &$c_j$&	4&	1&	5&	0&	0&&\\
    \midrule
    $c_B$&	Basis&	\cellcolor{blue!20}$y_1$&	$y_2$&	$y_3$&	$s_1$&	$s_2$&	$\pmb{b}$&\\
    \midrule
0&	\cellcolor{blue!20}$s_1$&	3&	-1&	0&	1&	-1&	15 &\cellcolor{blue!20} \textrightarrow\\
5&	$y_3$&	\nicefrac{3}{5}&	\nicefrac{4}{5}&	1&	0&	\nicefrac{1}{5}&	6&\\
\midrule
	&$Z_j$&	3&	4&	5&	0&	1&	30&\\
      &$c_j- Z_j$&	1&	-3&	0&	0&	-1&&\\
      &&\cellcolor{blue!20}\textuparrow&&&&&&\\
        \toprule
  \end{tabular}
  \par}

The new optimal table is

    {\centering
    \begin{tabular}{rrrrrrrrr}
    \toprule
    &$c_j$&	4&	1&	5&	0&	0&&\\
    \midrule
    $c_B$&	Basis&	$y_1$&	$y_2$&	$y_3$&	$s_1$&	$s_2$&	$\pmb{b}$&\\
    \midrule
4&	$y_1$&	1&	\nicefrac{-1}{3}&	0&	\nicefrac{1}{3}&	\nicefrac{-1}{3}&	5 & \\
5&	$y_3$&	0&	1&	1&	\nicefrac{-1}{5}&	\nicefrac{2}{5}&	3&\\
\midrule
	&$Z_j$&	4&	\nicefrac{11}{3}&	5&	\nicefrac{1}{3}&	\nicefrac{2}{3}&	35&\\
      &$c_j- Z_j$&	\cellcolor{blue!20}0&\cellcolor{blue!20}	\nicefrac{-8}{3}&\cellcolor{blue!20}	0&\cellcolor{blue!20}	\nicefrac{-1}{3}&\cellcolor{blue!20}	\nicefrac{-2}{3}&&\\
        \toprule
  \end{tabular}
  \par}

with $y_1, y_3$ in the basis. The new value of the objective function is 35.
  \end{solution}
  \part If additional 10 units of raw material can be obtained at a cost of \$ 12, is it profitable to do so?
  \begin{solution}{}
    
    $\pmb{B^{-1}} =
    \begin{bmatrix}
      1 & -1\\
      0 & \nicefrac{1}{5}
    \end{bmatrix}
    $
    and the new right-hand side vector is $ \pmb{b} = 
    \begin{bmatrix}
      45\\
      40
    \end{bmatrix}
    $  then the new solution is

    $
\begin{bmatrix}
  s_1 \\
  y_3
\end{bmatrix} = \pmb{B^{-1} \cdot b} = %
 \begin{bmatrix}
      1 & -1\\
      0 & \nicefrac{1}{5}
    \end{bmatrix}
    \begin{bmatrix}
      45\\
      40
    \end{bmatrix} = %
    \begin{bmatrix}
      5\\
      8
    \end{bmatrix}
    $
    the solution is feasible and the new value of the objective function is $Z = c_B \cdot %
\begin{bmatrix}
  s_1\\
  y_3
\end{bmatrix} = 
\begin{bmatrix}
  0 & 5
\end{bmatrix}
\begin{bmatrix}
  5\\
  8
\end{bmatrix} = \$40$.

As you can see, it is not profitable increment the resource in the raw material constraint. You only get an increment of \$10 versus the cost of increment the second resource that is \$12 per 10 additional units resource of raw material. Get the performance of the first value in the objective function against the second objective function value when you increment the resource in the raw material constraint, i.e.  $40 - 30$.
\end{solution}

\part If the available raw material is increased to 50 units, what is the optimal solution?

\begin{solution}{}

      $
\begin{bmatrix}
  s_1 \\
  y_3
\end{bmatrix} = \pmb{B^{-1} \cdot b} = %
 \begin{bmatrix}
      1 & -1\\
      0 & \nicefrac{1}{5}
    \end{bmatrix}
    \begin{bmatrix}
      45\\
      50
    \end{bmatrix} = %
    \begin{bmatrix}
      -5\\
      10
    \end{bmatrix}
    $
    the solution is infeasible. Apply the \textbf{dual simplex technique} and get the new feasible optimal solution.
    
{\centering
  \begin{tabular}{rrrrrrrrr}
    \toprule
    &$c_j$&	3&	1&	5&	0&	0&&	\\
    \cmidrule{2-7}
    $c_B$&	\textbf{Basis}&	$y_1$&	$y_2$&	$y_3$&	$s_1$&	\cellcolor{orange!50}$s_2$&	$\pmb{b}$&\\
    \midrule
0&	\cellcolor{orange!50}$s_1$&	3&	$-1$&	0&	1&	$-1$&	$-5$&\textrightarrow\\
    5&	$y_3$&	   3/5&   	   4/5&   	1&	0&	   1/5&   	10&\\
    \midrule
	&$Z_j$&	3&	4&	5&	0&	1&	50&\\
    &$c_j- Z_j$&	0&	$-3$&	0&	0&	$-1$&&	\\
    &&&&&&\cellcolor{orange!50}\textuparrow&&\\
    \toprule
    &ratios&&3&&&1&&\\
    \end{tabular}
    \par}

  The new optimal solution is 
  
{\centering
  \begin{tabular}{rrrrrrrrr}
    \toprule
    &$c_j$&	3&	1&	5&	0&	0&&	\\
    \cmidrule{2-7}
    $c_B$&	\textbf{Basis}&	$y_1$&	$y_2$&	$y_3$&	$s_1$&	$s_2$&	$\pmb{b}$&\\
    \midrule
0&	$s_2$&	$-3$&	1&	0&	$-1$&	1&	5&\\
    5&	$y_3$&	   6/5&   	   3/5&   	1&	1/5&	   0&   	9&\\
    \midrule
	&$Z_j$&	6&	3&	5&	1&	0&	\cellcolor{yellow}45&\\
	&$c_j- Z_j$&	$-3$&	$-2$&	0&	$-1$&	0&&	\\
    \toprule
    \end{tabular}
    \par}
  
  \end{solution}
  \part Due to ``technological breakthrough'' the raw material required by part B is reduced to 2 units. Will affect the optimal solution?
  \begin{solution}{}

    It will not affect. $x_2$ is a non-basic variable. The simplex multipliers are \[ \pmb{\pi} = c_B \cdot B^{-1} = \begin{bmatrix}
        0 & 5
      \end{bmatrix}
      \begin{bmatrix}
      1 & -1\\
      0 & \nicefrac{1}{5}
    \end{bmatrix} =
    \begin{bmatrix}
      0 & 1
    \end{bmatrix}
\]

    then \[ \overline{c_2} = c_2 - Z_2 = c_2 - \pmb{\pi} P_2 = 1 -
      \begin{bmatrix}
        0 & 1
      \end{bmatrix}
    \begin{bmatrix}
      3\\
      2
    \end{bmatrix} = 1 - 2 = -1
  \] Since $\overline{c_2} \leq 0$ remains non-positive in the solution, therefore it will not affect the solution.

  In general, \[ \overline{c_j} = c_j - Z_j = c_j - c_B \cdot B^{-1} \cdot P_j = 1 -
      \begin{bmatrix}
        0 & 5
      \end{bmatrix}
      \begin{bmatrix}
      1 & -1\\
      0 & \nicefrac{1}{5}
    \end{bmatrix}
    \begin{bmatrix}
      3\\
      2
    \end{bmatrix} = 1 - 2 = -1
  \]
    
  \end{solution}
  \part If a supervision constraint, $2y_1 + y_2 + 3y_3 \leq 20$ is added to the original problem, how is the optimal solution affected?
  \begin{solution}

    The optimal solution is not affected, the optimal solution is $s1 = 15$ and $y_3 = 6$, then when you add the constraint $2y_1 + y_2 + 3y_3 \leq 20$, and use the optimal solution we get \[ 2(0) + (0) + 3(6) = 18 \leq 20 \] therefore, the constraint is satisfied by the current optimal solution, so the optimal solution is not affected.
  \end{solution}
\end{parts}
\end{questions}
\end{document}